\documentclass{beamer}
\usetheme{Berkeley}

\usepackage{default}

\title[Pathfinder] % (optional, only for long titles)
{PATHFINDER}
\subtitle{Storage and Indexing of Massive Trajectory Sets}
\author[Funke, Nusser, Rupp, Storandt] % (optional, for multiple authors)
{S.~Funke\inst{1} \and A.~Nusser\inst{2} \and T.~Rupp\inst{3} \and S.~Storandt\inst{4}}
\institute[Universities] % (optional)
{
	\inst{1}%
	University of Stuttgart
	\and
	\inst{2}%
	Max Planck Institute for Informatics
	\and
	\inst{3}%
	University of Stuttgart
	\and
	\inst{4}%
	University of Konstanz
}
\date[SSTD 2019] % (optional)
{SSTD, 2019}
\subject{Computer Science}


\begin{document}
\frame{\titlepage}

\frame{\tableofcontents}

\section{Problem Description}
\begin{frame}
	\frametitle{Problem Description}
	\begin{itemize}
		\item<1-> road network graph $G(V,E,c)$
		\item<2-> trajectory data is provided as a collection $\mathcal{T}$
		\item<3-> $t\in \mathcal{T}$ is a sequence of nodes $\pi=v_0 v_1 \dots v_k$ in  $G$ annotated with timestamps $\tau_0, \tau_1, \dots, \tau_k$.
		\item<4-> \emph{window}-query of the form $[x_l, x_u]\times[y_l, y_u]\times[\tau_l, \tau_u]$
		\item<5-> goal: Return all intersecting trajectories
	\end{itemize}
\end{frame}

\section{Basics}
\begin{frame}
	\frametitle{Contraction Hierarchy}
	\begin{itemize}
		\item<1-> CH is a augmentation of a graph $G(V,E,c)$ with \emph{shortcuts} and \emph{node levels}.
		\item<2-> \emph{node contraction}: removes a node $v$ and all of its adjacent edges from the graph
		\item<3-> To maintain shortest path distances in the graph: Create shortcut $s = (u, w)$ between two adjacent nodes $u,w$ of $v$ if the only shortest path from $u$ to $w$ is the path $uvw$. $c(s) = c(uv) + c(vw)$.
		\item<4-> Construct CH by successively contracting all nodes
		\item<5-> final CH data structure is defined as $G(V, E^+, c, l)$ where $E^+$ is the union of $E$ and all shortcuts
	\end{itemize}
\end{frame}

\begin{frame}
	\frametitle{Inverted Index}
\end{frame}

\begin{frame}
	\frametitle{Shortcut Representation}
\end{frame}

\begin{frame}
	\frametitle{Shortcut Representation}
	\framesubtitle{Compression}
\end{frame}

\begin{frame}
	\frametitle{CH as RTree}
\end{frame}

\section{Algorithm}

\begin{frame}
	\frametitle{Algorithm Overview}
\end{frame}

\begin{frame}
	\frametitle{FindEdgeCandidates}
\end{frame}

\begin{frame}
	\frametitle{RefineEdgeCandidates}
\end{frame}

\begin{frame}
	\frametitle{GetAssociatedTrajectories}
\end{frame}

\begin{frame}
	\frametitle{TimeFrameQueries}
\end{frame}

\begin{frame}
	\frametitle{Parallelization}
\end{frame}

\section{Experiments}

\begin{frame}
	\frametitle{OnDisk (SSTD)}
\end{frame}

\begin{frame}
	\frametitle{Experiments}
	\framesubtitle{Hardware}
\end{frame}

\begin{frame}
	\frametitle{Experiments}
	\framesubtitle{Data}
\end{frame}

\begin{frame}
	\frametitle{Experiments}
	\framesubtitle{Evaluations compression}
\end{frame}

\begin{frame}
	\frametitle{Experiments}
	\framesubtitle{Evaluations retrieval}
\end{frame}
% etc
\end{document}
